\documentclass{article}
\usepackage{amsmath}
\usepackage{hyperref}
\usepackage[a4paper,margin=2.5cm]{geometry}
\usepackage{titlesec}


\title{Notes on Cycloidal Drives}
\author{Drew Imhof}
\date{\today}

\begin{document}

\maketitle

\section{Design of Cycloidal Disc}
\subsection{Cycloidal Disc}

\begin{itemize}
    \item Cycloidal shape obtained by rolling circle that rolls around a base circle
    \item Cycloid corresponds to path described by a point at the circumference of the rolling circle (ordinary cycloid)
    \item Diameter of fixed pins will correspond w/ diameter of drawing circle used to actually make cycloid
    \item Will likely want contracted cycloid to decrease eccentricity and unbalanced forces at high speeds (r $<$ R)
    \item Due to symmetrical load distribution, two cycloidal discs are often used and offset by 180\textdegree\
\end{itemize}

\subsection{Transmission Ratio}
\begin{align}
    Transmission Ratio = i = \frac{n}{N - n} \\
    where, N = 1 + n
\end{align}

\begin{itemize}
    \item \textit{n} = \# of rotor teeth
    \item \textit{N} = \# of rollers
\end{itemize}

\begin{align}
    i = \frac{d}{\delta} \\
    \delta = \frac{D}{N} \\ 
    % i = \frac{d}{\delta} = \frac{d*N}{D} \\
    d = \frac{i}{N}D
\end{align}

\begin{itemize}
    \item \textit{$\delta$} is the diameter of the rolling circle
    \item \textit{d} is base circle diameter
    \item \textit{D} is the pitch circle of the fixed diameter pins
    \item \textit{N} is the \# of fixed pins
\end{itemize}

\subsection{Eccentricity}
\begin{equation}
    e \le \frac{\delta}{2}
\end{equation}

\begin{itemize}
    \item If \textit{e} is too small, the shape of the cycloid will become too soft and get close to a circle
    \item This can cause increasing backlash
\end{itemize}

\subsection{Hole diameter of the cycloidal disc}
\begin{equation}
    d_h = d_r +2e
\end{equation}

\begin{itemize}
    \item \textit{$d_h$} is the diameter of holes in the cycloidal disc
    \item \textit{$d_r$} is the diameter of the roller pins 
\end{itemize}

\section{Epitrochoid}
An epitrochoid is a roulette traced by a pt attached to a circle of radius \textit{r}
rolling around the outside of a fixed circle of radius \textit{R}. The point is at a distance
\textit{d} from the center of the exterior circle. The parametric equations are as follows:

\begin{align}
    x(\theta) = (R+r)\cos\theta - d\cos(\frac{R+r}{r}\theta) \\
    y(\theta) = (R+r)\sin\theta - d\sin(\frac{R+r}{r}\theta)
\end{align}

\section{References}

\begin{enumerate}
    \item  \href{https://www.tec-science.com/mechanical-power-transmission/planetary-gear/construction-of-the-cycloidal-disc/}{Construction of Cycloidal Disc}
    \item  \href{https://digitalcommons.njit.edu/cgi/viewcontent.cgi?article=2822&context=theses}{Thesis involoving cycloidal drives}
    \item \href{https://www.researchgate.net/publication/235992854_A_New_Design_of_a_Two-Stage_Cycloidal_Speed_Reducer}{Thesis with Dual Stage cycloidal drive}
    \item  \href{https://howtomechatronics.com/how-it-works/what-is-cycloidal-driver-designing-3d-printing-and-testing/}{Cycloidal drive design and printing}
    \item  \href{https://blogs.solidworks.com/teacher/wp-content/uploads/sites/3/Building-a-Cycloidal-Drive-with-SOLIDWORKS.pdf}{Cycloidal drive in Solidworks}
    \item  \href{https://stepbystep-robotics.com/hp/robots/cycloidal-drive/}{stepbystep-robotics guide}
    \item  \href{https://github.com/roTechnic/CycloidalDesign/blob/main/pyplot\%20cycloid.py}{Useful cycloid github}
    \item  \href{https://en.wikipedia.org/wiki/Epitrochoid}{Wiki page on Epitrochoid}
    \item  \href{https://en.wikipedia.org/wiki/Epicycloid}{Wiki page on epicycloid}
\end{enumerate}

\end{document}